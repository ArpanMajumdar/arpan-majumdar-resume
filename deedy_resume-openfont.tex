%%%%%%%%%%%%%%%%%%%%%%%%%%%%%%%%%%%%%%%
% Deedy - One Page Two Column Resume
% LaTeX Template
% Version 1.2 (16/9/2014)
%
% Original author:
% Debarghya Das (http://debarghyadas.com)
%
% Original repository:
% https://github.com/deedydas/Deedy-Resume
%
% IMPORTANT: THIS TEMPLATE NEEDS TO BE COMPILED WITH XeLaTeX
%
% This template uses several fonts not included with Windows/Linux by
% default. If you get compilation errors saying a font is missing, find the line
% on which the font is used and either change it to a font included with your
% operating system or comment the line out to use the default font.
% 
%%%%%%%%%%%%%%%%%%%%%%%%%%%%%%%%%%%%%%
% 
% TODO:
% 1. Integrate biber/bibtex for article citation under publications.
% 2. Figure out a smoother way for the document to flow onto the next page.
% 3. Add styling information for a "Projects/Hacks" section.
% 4. Add location/address information
% 5. Merge OpenFont and MacFonts as a single sty with options.
% 
%%%%%%%%%%%%%%%%%%%%%%%%%%%%%%%%%%%%%%
%
% CHANGELOG:
% v1.1:
% 1. Fixed several compilation bugs with \renewcommand
% 2. Got Open-source fonts (Windows/Linux support)
% 3. Added Last Updated
% 4. Move Title styling into .sty
% 5. Commented .sty file.
%
%%%%%%%%%%%%%%%%%%%%%%%%%%%%%%%%%%%%%%%
%
% Known Issues:
% 1. Overflows onto second page if any column's contents are more than the
% vertical limit
% 2. Hacky space on the first bullet point on the second column.
%
%%%%%%%%%%%%%%%%%%%%%%%%%%%%%%%%%%%%%%


\documentclass[]{deedy-resume-openfont}
\usepackage{fancyhdr}
 
\pagestyle{fancy}
\fancyhf{}
\tolerance=1
\emergencystretch=\maxdimen
\hyphenpenalty=10000
\hbadness=10000
 
\begin{document}

%%%%%%%%%%%%%%%%%%%%%%%%%%%%%%%%%%%%%%
%
%     LAST UPDATED DATE
%
%%%%%%%%%%%%%%%%%%%%%%%%%%%%%%%%%%%%%%
\lastupdated

%%%%%%%%%%%%%%%%%%%%%%%%%%%%%%%%%%%%%%
%
%     TITLE NAME
%
%%%%%%%%%%%%%%%%%%%%%%%%%%%%%%%%%%%%%%
\namesection{}{Arpan Majumdar}{ \urlstyle{same}
Email: \href{mailto:arpan.majumdar.dev@gmail.com }{arpan.majumdar.dev@gmail.com } |     Mobile: +917974287080 \\ 
LinkedIn: \href{https://www.linkedin.com/in/arpan-majumdar-dev}{https://www.linkedin.com/in/arpan-majumdar-dev} \\
Github: \href{https://github.com/ArpanMajumdar}{https://github.com/ArpanMajumdar}
}

%%%%%%%%%%%%%%%%%%%%%%%%%%%%%%%%%%%%%%
%
%     COLUMN ONE
%
%%%%%%%%%%%%%%%%%%%%%%%%%%%%%%%%%%%%%%

\begin{minipage}[t]{0.33\textwidth} 

%%%%%%%%%%%%%%%%%%%%%%%%%%%%%%%%%%%%%%
%     SKILL SET
%%%%%%%%%%%%%%%%%%%%%%%%%%%%%%%%%%%%%%

\section{Skill set} 
\subsection{Languages}
\textbf{Proficient:} Java, Python \\
\textbf{Familiar:} Scala, Kotlin
\sectionsep

\subsection{Frameworks}
• Springboot and related libraries (Spring batch, Spring state machine, Spring kafka, Spring Security) \\
• Micronaut
\sectionsep

\subsection{Technologies}
• Apache Spark (Big Data) \\
• Docker (Containerization) \\
• Drone (CI/CD) \\
• Kafka (Message broker) \\
• Redis (Caching)
\sectionsep

\subsection{Testing}
• JUnit4/5 \\
• Mockito \\
• Powermockito
\sectionsep

\subsection{Databases}
• Postgres \\
• Mongo \\
• Druid
\sectionsep

\subsection{Other skills}
• Image processing \\
• Machine learning \\
• Web development
\sectionsep

%%%%%%%%%%%%%%%%%%%%%%%%%%%%%%%%%%%%%%
%     EDUCATION
%%%%%%%%%%%%%%%%%%%%%%%%%%%%%%%%%%%%%%
\section{Education}
\subsection{Undergraduate}
\descript{NIT WARANGAL | AUG 2013 – JUNE 2017}
\location{B.TECH IN ELECTRICAL AND ELECTRONICS ENGINEERING} 
CGPA on a scale of 10 : \textbf{8.61}
\sectionsep

\subsection{Higher Secondary CBSE}
\descript{SRI CHAITANYA TECHNO SCHOOL, VISHAKHAPATNAM, ANDHRA PRADESH | JUNE 2012 – MAY 2013}
Percentage: \textbf{95}
\sectionsep



%%%%%%%%%%%%%%%%%%%%%%%%%%%%%%%%%%%%%%
%
%     COLUMN TWO
%
%%%%%%%%%%%%%%%%%%%%%%%%%%%%%%%%%%%%%%

\end{minipage} 
\hfill
\begin{minipage}[t]{0.66\textwidth} 

%%%%%%%%%%%%%%%%%%%%%%%%%%%%%%%%%%%%%%
%     EXPERIENCE
%%%%%%%%%%%%%%%%%%%%%%%%%%%%%%%%%%%%%%

\section{Work Experience}
\runsubsection{TARGET CORPORATION}
\descript{| Software Engineer }
\location{ JULY 2017 - PRESENT | Bengaluru}
Nearly 2.5 years of work experience in Retail Industry in the Merchandising and Search Teams in Target Corporation.
\sectionsep

\section{Work Projects}
\runsubsection{TARGET VENDOR INCOME}
\location{| JULY 2017 - JUNE 2019}
\vspace{\topsep}
\begin{tightemize}
\item Worked as a back-end developer for web app which helps Target collect Vendor Income by making contracts and deals with the contract negotiation process.
\item The application uses a \textbf{React UI}, \textbf{SpringBoot backend} and \textbf{Postgres database}.
\item It also feeds upstream and downstream consumers using \textbf{Kafka} messaging service.
\end{tightemize}
\sectionsep

\runsubsection{SEARCH RELEVANCY ANALYTICS TEAM}
\location{| JULY 2019 - PRESENT}
\vspace{\topsep} % Hacky fix for awkward extra vertical space
\begin{tightemize}
\item Worked in the search analytics team to gain insights from Target Search data and calculate metrics like \textbf{Click Through Rate (CTR)}, \textbf{Relevancy}, \textbf{Profitability}, \textbf{Null Rate} etc. Results of the analytics are used by various teams to improve their products and to catch anomalies.
\item Designed \textbf{scalable data pipelines} for ingesting and progessing data with throughput of \textbf{3000 TPS(avg)/15000 TPS (peak)}.
\item Used \textbf{Apache Spark} and \textbf{Kafka streams} for data analysis and \textbf{druid database} for storing and power dashboards from calculated metrics.
\end{tightemize}
\sectionsep

%%%%%%%%%%%%%%%%%%%%%%%%%%%%%%%%%%%%%%
%     PERSONAL PROJECTS
%%%%%%%%%%%%%%%%%%%%%%%%%%%%%%%%%%%%%%

\section{Personal Projects}

\runsubsection{MELT FRACTION CALCULATION OF PCM USING IMAGE PROCESSING}
\location{| Feb 2017 – June 2018}
\begin{tightemize}
\item This was a research project aimed to \textbf{calculate melt fraction of a phase change material (PCM) using image processing} which was otherwise needed to be calculated using SolidWorks. This reduced the calculation time from \textbf{10-15 min/image} to less than \textbf{1sec/image}, thus improving the speed of research.
\item Images of melting PCM were taken at regular intervals. \textbf{Image segmentation} and \textbf{3-D rotation} were performed to extract solid and liquid regions.
\item A \textbf{genetic algorithm} was used for circle fitting and \textbf{K-Means} for image segmentation.
\end{tightemize}
\sectionsep

\runsubsection{COMPUTATIONAL INTELLIGENCE LAB, AEROSPACE DEPARTMENT, INDIAN INSTITUTE OF SCIENCE, BANGALORE UNDER PROF. S. N. OMKAR }
\location{| MAY 2016 - JULY 2016}
\begin{tightemize}
\item Aim of the project is to \textbf{perform mineral mapping of lunar hyperspectral data} obtained from \textbf{Moon Mineralogy Mapper(M3)} onboard \textbf{Chandrayaan-I}.
\item \textbf{Fuzzy K-means} and \textbf{Expectation Maximization} clustering algorithms are used to cluster similar spectra and \textbf{Support Vector Machine(SVM)} was used for classification and comparison with lunar library spectra.
\end{tightemize}
\sectionsep

\end{minipage}

\begin{minipage}[t]{0.33\textwidth} 
%%%%%%%%%%%%%%%%%%%%%%%%%%%%%%%%%%%%%%
%     MOOCs
%%%%%%%%%%%%%%%%%%%%%%%%%%%%%%%%%%%%%%
\section{MOOCs}
\descript{Machine Learning}
\location{by Andrew Ng}
\sectionsep
\descript{DeepLearning.ai specialization}
\location{by Andrew Ng}
\sectionsep
\descript{PYTHON FOR DATA SCIENCE}
\location{ by University of Washington}
\sectionsep
\end{minipage}
\hfill
\begin{minipage}[t]{0.66\textwidth} 
\runsubsection{FLOORPLAN TO JSON PROJECT UNDER AAPKA PAINTER STARTUP  }
\location{| JAN 2016 - MAR 2016}
\begin{tightemize}
\item Aim of the project is to \textbf{convert floor plan given in image format to description of walls and their arrangement in JSON format} so that the JSON file can be used as input for \textbf{Aapka Painter 3d rendering software}.
\item \textbf{Hough transform} for line detection and \textbf{Harris Corner Detection} algorithms are used to identify walls and corners respectively in the floor plan.
\end{tightemize}
\sectionsep

\runsubsection{GAME AUTOMATION - ONE TOUCH DRAW   }
\location{| JAN 2016}
\begin{tightemize}
\item This project aims at \textbf{hacking the Android Game ”One Touch Draw”}. The Connection to PC is either wireless or using USB and game is played on its own with ADB tools.
\item One touch draw is a simple \textbf{Eular’s path} graph problem in which player has to traverse the figure without lifting his/her finger.
\end{tightemize}
\sectionsep

\section{Achievements} 
\vspace{\topsep}
\begin{tightemize}
\item Paper titled \textbf{\href{https://onlinelibrary.wiley.com/doi/10.1002/er.4668}{"An image processing algorithm to estimate the melt fraction and energy storage of a PCM enclosed in a spherical capsule"}} is published in \textbf{International Journal of Energy Research. \href{https://onlinelibrary.wiley.com/doi/10.1002/er.4668}{DOI: 10.1002/er.4668}}
\item Paper titled \textbf{”A spectral-spatial method for mineral mapping of M3 data using multiple classification approach”} is under review.
\item Paper titled \textbf{\href{https://www.researchgate.net/publication/310812213_Drawbot_A_Mobile_Robot_for_Image_Scanning_and_Scaled_Printing}{Drawbot: A Mobile Robot for Image Scanning and Scaled Printing}} published by \textbf{International Journal of Mechanical Engineering and Robotics Research.. 5. 10.18178/ijmerr.5.2.124-128.}
\item Received merit scholarship in 1st year.
\end{tightemize}
\sectionsep
\end{minipage}

\end{document}  \documentclass[]{article}
